% Options for packages loaded elsewhere
\PassOptionsToPackage{unicode}{hyperref}
\PassOptionsToPackage{hyphens}{url}
%
\documentclass[
]{article}
\title{The Fundamental Law of Road Congestion:Evidence from US cities}
\usepackage{etoolbox}
\makeatletter
\providecommand{\subtitle}[1]{% add subtitle to \maketitle
  \apptocmd{\@title}{\par {\large #1 \par}}{}{}
}
\makeatother
\subtitle{Final Deliverable Reproduction Package}
\author{Julian Naranjo (201921367) y Christian André Villegas
(201731959)}
\date{29 de mayo, 2022}

\usepackage{amsmath,amssymb}
\usepackage{lmodern}
\usepackage{iftex}
\ifPDFTeX
  \usepackage[T1]{fontenc}
  \usepackage[utf8]{inputenc}
  \usepackage{textcomp} % provide euro and other symbols
\else % if luatex or xetex
  \usepackage{unicode-math}
  \defaultfontfeatures{Scale=MatchLowercase}
  \defaultfontfeatures[\rmfamily]{Ligatures=TeX,Scale=1}
\fi
% Use upquote if available, for straight quotes in verbatim environments
\IfFileExists{upquote.sty}{\usepackage{upquote}}{}
\IfFileExists{microtype.sty}{% use microtype if available
  \usepackage[]{microtype}
  \UseMicrotypeSet[protrusion]{basicmath} % disable protrusion for tt fonts
}{}
\makeatletter
\@ifundefined{KOMAClassName}{% if non-KOMA class
  \IfFileExists{parskip.sty}{%
    \usepackage{parskip}
  }{% else
    \setlength{\parindent}{0pt}
    \setlength{\parskip}{6pt plus 2pt minus 1pt}}
}{% if KOMA class
  \KOMAoptions{parskip=half}}
\makeatother
\usepackage{xcolor}
\IfFileExists{xurl.sty}{\usepackage{xurl}}{} % add URL line breaks if available
\IfFileExists{bookmark.sty}{\usepackage{bookmark}}{\usepackage{hyperref}}
\hypersetup{
  pdftitle={The Fundamental Law of Road Congestion:Evidence from US cities},
  pdfauthor={Julian Naranjo (201921367) y Christian André Villegas (201731959)},
  hidelinks,
  pdfcreator={LaTeX via pandoc}}
\urlstyle{same} % disable monospaced font for URLs
\usepackage[margin=1in]{geometry}
\usepackage{longtable,booktabs,array}
\usepackage{calc} % for calculating minipage widths
% Correct order of tables after \paragraph or \subparagraph
\usepackage{etoolbox}
\makeatletter
\patchcmd\longtable{\par}{\if@noskipsec\mbox{}\fi\par}{}{}
\makeatother
% Allow footnotes in longtable head/foot
\IfFileExists{footnotehyper.sty}{\usepackage{footnotehyper}}{\usepackage{footnote}}
\makesavenoteenv{longtable}
\usepackage{graphicx}
\makeatletter
\def\maxwidth{\ifdim\Gin@nat@width>\linewidth\linewidth\else\Gin@nat@width\fi}
\def\maxheight{\ifdim\Gin@nat@height>\textheight\textheight\else\Gin@nat@height\fi}
\makeatother
% Scale images if necessary, so that they will not overflow the page
% margins by default, and it is still possible to overwrite the defaults
% using explicit options in \includegraphics[width, height, ...]{}
\setkeys{Gin}{width=\maxwidth,height=\maxheight,keepaspectratio}
% Set default figure placement to htbp
\makeatletter
\def\fps@figure{htbp}
\makeatother
\setlength{\emergencystretch}{3em} % prevent overfull lines
\providecommand{\tightlist}{%
  \setlength{\itemsep}{0pt}\setlength{\parskip}{0pt}}
\setcounter{secnumdepth}{5}
\usepackage[spanish,es-tabla]{babel}
\usepackage[utf8]{inputenc}
\usepackage{fancyhdr}
\pagestyle{fancy}
\setlength\headheight{50pt}
\usepackage{float}
\floatplacement{figure}{H}
\usepackage{booktabs}
\usepackage{longtable}
\usepackage{array}
\usepackage{multirow}
\usepackage{wrapfig}
\usepackage{float}
\usepackage{colortbl}
\usepackage{pdflscape}
\usepackage{tabu}
\usepackage{threeparttable}
\usepackage{threeparttablex}
\usepackage[normalem]{ulem}
\usepackage{makecell}
\usepackage{xcolor}
\ifLuaTeX
  \usepackage{selnolig}  % disable illegal ligatures
\fi

\begin{document}
\maketitle

\setcounter{tocdepth}{5}
\renewcommand\thesection{\roman{section}}

\hypertarget{abstract}{%
\section{Abstract}\label{abstract}}

El propósito principal del paper ``The Fundamental Law of Road
Congestion: Evidence from US Cities'', donde la congestión fue medida
como los vehículos por kilómetros recorrido (VKT por sus siglas en
inglés), se puede dividir en 3 objetivos. En primer lugar, busca estimar
la elasticidad de la demanda que tiene las carretas nuevas en la
congestión de cada metropolitan statistical áreas (MSA), durante los
años 1983, 1993, y 2003. En segundo lugar, se busca determinar si se
cumple la ley de congestión de vías, en donde se encuentra que el
transporte público no tiene un efecto sobre las congestión y que existe
un equilibrio de ``Tráfico'', el cual consiste en que cada ciudad tiene
un tráfico ``optimo''. En este paper los autores logran determinar que
en el tiempo la congestión va disminuyendo. Finalmente, los autores
determinan los mecanismos por los cuales la congestión aumenta ante
incrementos del tráfico, para esto estiman un modelo que está en función
del variable de comercio, el crecimiento poblacional y la disponibilidad
de carreteras.

\hypertarget{introducction}{%
\section{Introducction}\label{introducction}}

Para la estimación, los autores tienen 4 formas funcionales en su
cabeza, la primera es una regresión lineal, estimada por ordinary least
squares OLS, donde se viola el supuesto de exógeneidad, dado que pude
existir un ``cofunder'', el cual aumente VKT y las carreteras, como por
ejemplo mayor población. Para intentar corregir esto, se crea la
ecuación 2, la cual añade efectos fijos de las metropolitan statistical
áreas (MSA), incluyendo dummys por cada una, estimando un pool OLS. Otra
forma de resolver los problemas de contar con ``cofunder'', es realizar
la primera diferencia, no obstante, los efectos fijos por MSA
desaparecen, para esto se añade puntualmente condiciones climáticas y
socioeconómicas de la población, estimando mediante pool OLS.

\begin{equation}
    ln(Q_{i,t})= A_0 + \alpha ln(R_{i,t})+ A_1 X_{i,t}+ \epsilon_{i,t}
\end{equation}

\begin{equation}
    ln(Q_{i,t})= A_0 + \alpha ln(R_{i,t})+ A_1 X_{i,t}+ \delta_i +\epsilon_{i,t}
\end{equation}

\begin{equation}
    \triangle ln(Q_{i,t})=  \alpha \triangle ln(R_{i,t})+ A_1 \triangle X_{i,t}+ \triangle \epsilon_{i,t}
\end{equation}

\begin{align}
\triangle ln(R_{i,t})=  B_0 + B_1 X_{i,t} + B_2 Z_{i,t}+ \mu_{i,t} \\
\triangle ln(Q_{i,t})=  A_0 + \alpha ln(R_{i,t})+ A_1 X_{i,t}+ \epsilon_{i,t}
\end{align}

Los autores reconocen que la fuente de endógenidad proviene de la existe
una relación de doble causalidad entre VKT y unidades adicionales de
carreteras, dado que más congestión incentiva a construir más carretas,
dando como resultado mayor tráfico, mientras que mayor carreteras
generan mayor uso de estas. Para esto, se implemente una estimación de
variables instrumentales. Los tres instrumentos planteados fueron: 1)
las rutas utilizadas por los exploradores de américa durante 1835 y
1850; 2) el plan de autopista de 1947 para mejorar el comercio de EE.UU;
3) las antiguas rutas del tren. Consideramos que los tres instrumentos
cumplen el criterio de ser ``fuertes'', dado que logran explicar el VKT
únicamente por medio del aumento de carreteras, las cuales todas
utilizaron esta base metodología para hacerse.

Todas las anteriores estimaciones, fueron presentadas en 8 tablas,
sumadas a 2 tablas descriptivo de las variables utilizadas y 2 gráficos,
que muestran cómo se relacionan las carreteras de instrumentos con las
carreteras actuales en EEUU. Por otro lado, la estimación preferida por
los investigadores es la que fue estimada utilizando el método de IV,
con los 3 instrumentos y controlando por variables físicas de geografía
y algunas variables del censo para los MSA. Consideramos que es una
buena elección, dado que se logra controlar la endogenidad por el frente
de doble causalidad de las variables y características propias de los
MSA, los cuales pueden actuar como cofunder. Los test de robustes, son
pocos, en su mayoría es cambiar la especificación de los años evaluados
y las carreteras seleccionadas para el análisis. No obstante cuentan con
la ventaja de que las estimaciones de la elasticidades son muy
parecidas, bajo todas las medidas. Ante esto, los autores, muestran que
pude existir que choques negativos de la zona, causan mayores
carreteras, el cual es un resultado que nosotros quisiéramos validar,
dado que nos parece contraituitivo.

Los datos utilizados, los MSA parecen estándares de EEUU, dado que los
autores no determinan cómo se construyen, los toman como dados. Respecto
a las carreteras, información del tamaño, ubicación de estas, vehículos
por tipo que transitan por ellas, se utilizan la información de US HPMS,
las cuales las ciudades deben reportante ante el Federal Highway
Administration in the US Department of Transportation (dot ), con el
cual se realizan los planes de inversión. Para la información a nivel
individual, se cuenta con información de NPTS, donde se recoge
información de distancia pro persona, edad, sexo y ocupación, durante un
periodo de 1995-2001.

\hypertarget{code-improvements}{%
\section{Code improvements}\label{code-improvements}}

En general, las mejoras al código que realizamos consistieron en sacar
de manera precisa todas las salidas desde Stata, sin necesidad de de
hacer pasos extras.

\begin{itemize}
\item
  Contribución 1: Las salidas correspondientes de Table 1, son recogidas
  por un documento .log, el cual guarda las salidas de la consola,
  siendo este documento un resultado intermedio con el que se construye
  la tabla. En general la mayoría de las variables están indexadas en el
  tiempo con los dos últimos dígitos del año (YY), correspondiente al
  final de cada variable (e.g.~ln\_pc\_IH83 corresponde al logaritmo de
  kilómetros de avenidas interestatales por cada 10.000 habitantes para
  el año 1983). Esta sección empieza entonces con el proceso de
  creación, escalamiento, o modificación de variables.
\item
  Contribución 2: Se genera el código para guardar directamente en LATEX
  la tabla 1, la cual desde su paquete de reproducción únicamente guarda
  algunos de los resultados requeridos para la construcción de la tabla
  por medio de un .log
\item
  Contribución 3: Se crean algunas variables que corresponde a las áreas
  no urbanas, este calculo se hace mediante el computo del resultado
  total menos el resultado urbano dentro de un logaritmo. Variable
  `Zero', corresponde al mínimo del logaritmo de los kilómetros de
  avenidas interestatales o avenidas principales urbanas para los
  componentes urbanos y no urbanos de cada MSA en todas las unidades de
  tiempo. Adicionalmente se hace un reescalamiento de algunas variables
  que serán usadas como controles en algunos ejercicios econométricos
  (cooling\_dd, heating\_dd, ruggedness\_msa, elevat\_range\_msa).
\item
  Contribucion 4:
\end{itemize}

Como lo indican los autores, en esta sección se crean los resultados de
la tabla 2 y 3 de manera conjunta, en el código nuevo elaborado estas
salidas salen automáticamente. Estas regresiones son compiladas en un
triple loop (k, j, i), donde k=\{1, 2, 3, 4\} indexa los paneles de las
tablas 2 y 3, j indexa el año de la regresión además de algunas de las
variables de carácter temporal a usar \{83, 93, 03\} y finalmente, i
indexa la especificación.

Respecto a K: 1: ln VKT for interstate highways, entire MSAs 2: ln VKT
for interstate highways, urbanized areas within MSAs 3: ln VKT for major
roads, urbanized areas within MSAs 4: ln VKT for interstate highways,
outside urbanized areas within MSAs

Respecto a i: 1: univariable 2: controlando por población 3: controlando
por población, variables geográficas y un efecto fijo por unidad censal.
4: controlando por población, variables geográficas, un efecto fijo por
unidad censal y variables que recogen el comportamiento poblaciónal y
socioeconómico de cada MSA.

Variable de salida: ln de los kilómetros viajados por vehículo (VKT)
Variable de interés: ln kilómetros de avenidas interestatales (IH)

Sobre la regresión en específico, esta es una regresión lineal por
mínimos cuadrados ordinarios (MSO) y no tiene ningún detalle especial a
resaltar, más allá que en las regresiones multivariable incluye la
variable `Zero', la cual garantiza que no se corra la regresión con
variables iguales a cero.

\begin{itemize}
\tightlist
\item
  Contribución 5: En esta sección se realiza estimaciones con modelos de
  regresión MCO Pooled, para lo cual es necesario homogeneizar la
  indexación de las variables y organizar la información disponible,
  i.e.~pasar la información de tipo `Wide' o tipo `Long', a
  continuación, se hace un pequeño resumen al respecto del proceso:
\end{itemize}

Al igual que en el módulo anterior de salidas, se generan las variables
correspondientes a las zonas no urbanizadas, y se reescalan variables
geográficas de cada MSA. Adicionalmente a la modificación de las
variables similar a la del módulo anterior se realiza un proceso de
homogenización de variables, en la consideración que se va a surtir un
proceso de transformación de la base de datos, para el caso se realiza
el ajuste de año (YY) de algunas variables considerando que hay un poco
rezago entre estas e.g.~S\_poor\_80 cambia su año por 1983. así mismo,
de esta misma manera se completa y se hacen expresiones regulares en la
base de datos para conseguir que cada variable (panel) de interés cuente
al final con el año respectivo SSYY, e.g.~S\_somecollege\_80 se
transforma a S\_somecollege\_1983.

Con las variables arregladas se procede a realizar la transformación de
la base a su correspondiente forma `long' (Panel), usando como variables
de pivote - individuo, el código de identificación de cada MSA; y como
indicador de Año, se crea la variable year que comprenderá los tres
periodos correspondientes 1983 1993 y 2003.

\begin{itemize}
\tightlist
\item
  Contribución 6: Teniendo en cuenta los pasos anteriores, se procese a
  estimar los modelos panel, cada uno de estos cuentan con
  especificaciones ligeramente diferentes que serán comentadas a
  continuación uno a uno.
\end{itemize}

Variable de salida: ln de los kilómetros viajados por vehículo (VKT)
Variable de interés: ln kilómetros de avenidas interestatales (IH)

Modelos estimados:

M1: Modelo panel con efectos fijos de tiempo, errores robustos y
clusterizados a nivel de MSA M2: M1 + Control por la población M3: M2 +
Controles de geografía y división censal M4: M3 + Controles de
características socioeconómicas y rezagos de la población M5: Modelo
Panel con efectos fijos de tiempo y MSA y con errores robustos M6: M5 +
Control por la población M7: M6 + Controles de características
socioeconómicas M8: M6, limitando la regresión a las zonas urbanas o
susceptibles a las avenidas interestatales, i.e.~que la variable `Zero'
sea estrictamente mayor a cero. M9: M6 para ciudades grandes,
considerando grandes a aquellas ciudades que para los 1990 contara con
más población que la mediana de las observaciones (l\_pop90
\textgreater{} 12.586090). M10: M6 para ciudades pequeñas (l\_pop90
\textless{} 12.586090).

\begin{itemize}
\tightlist
\item
  Contribución 7: Esta sección al igual que las anteriores, cuenta con
  una sección de creación y/o modificación de las variables, las cuales
  ya son recogidas por las explicaciones de los modulos anteriores.
\end{itemize}

En cuanto a las regresiones, esta sección se presentan 18 modelos de
regresión divididas en dos paneles, el panel A, que corresponde a
regresiones panel estimado con un modelo de MCO, y un segundo panel (B)
estimado a través de un modelo regresión de mínimos cuadrados en dos
etapas (Two Stage Least Squares - TSLS), la especificación de cada
modelo en el módulo A, se corresponde a uno en el B, menos para la
primera y novena estimación del módulo A. Así la especificación de
modelo A2, es equivalente a la del modelo B2, pero cambia la forma de
estimación, y así sucesivamente con los demás modelos.

Variable de salida: ln del cambio de kilómetros viajados por vehículo
(delta VKT) Variable de interés: ln del cambio de kilómetros de avenidas
interestatales (delta IH)

Modelos estimados:

M1: Modelo de primeras diferencias M2: M1 + Controles por población M3:
M2 + Control de los VKT en el periodo inicial de la información para
cada MSA. M4: M3 + Controles geográficos y de división censal M5: M4 +
Controles socioeconómicos y de rezagos de la población. M6: M2 para una
muestra restringida en la que se presenta un crecimiento de al menos un
5\% de las avenidas interestatales (Dl\_ln\_IH \textgreater0.05) M7: M5
para una muestra restringida en la que se presenta un crecimiento de al
menos un 5\% de las avenidas interestatales (Dl\_ln\_IH
\textgreater0.05) M8: M5 para una muestra restringida en la que se
presenta un decrecimiento de al menos un 5\% de las avenidas
interestatales (Dl\_ln\_IH \textless-0.05) M9: M1 + Efectos fijos por
MSA M10: M9 + Controles por población

Todas las regresiones cuentan con clusterizan los errores a nivel de
MSA, menos en los modelos 9 y 10, donde mediante efectos fijos se
recogen los efectos no observables. Y todas las estimaciones se hacen
con errores robustos.

\begin{itemize}
\tightlist
\item
  Contribución 8:
\end{itemize}

En esta sección se presentan 40 regresiones haciendo uso de variables
instrumentales. Precisamente el uso de estos instrumentos y la forma de
estimación serán recogido por cada uno de los paneles a tener en cuenta
(8), y diferentes especificaciones que se presentan para cada panel (5
por cada uno).

\begin{enumerate}
\def\labelenumi{\Alph{enumi}.}
\item
  Variable de salida: ln de los kilómetros viajados por vehículo (VKT)
  Variable de interés: ln kilómetros de avenidas interestatales (IH)
  Instrumentos: Rutas de exploración en 1835 (ln km), vías de
  ferrocarril en 1898 (ln km) y rutas interestatales planeadas en 1947
  (ln km) Modelo: MCO en dos etapas (TSLS)
\item
  Variable de salida: ln de los kilómetros viajados por vehículo (VKT)
  Variable de interés: ln kilómetros de avenidas interestatales (IH)
  Instrumentos: Rutas de exploración en 1835 (ln km), vías de
  ferrocarril en 1898 (ln km) y rutas interestatales planeadas en 1947
  (ln km) Modelo: Máxima verosimilutud con restricciones de información
  (Limited information maximum likelihood - LIML)
\item
  Variable de salida: ln de los kilómetros viajados por vehículo (VKT)
  Variable de interés: ln kilómetros de avenidas interestatales (IH)
  Instrumentos: Rutas interestatales planeadas en 1947 (ln km) Modelo:
  MCO en dos etapas (TSLS)
\item
  Variable de salida: ln de los kilómetros viajados por vehículo (VKT)
  Variable de interés: ln kilómetros de avenidas interestatales (IH)
  Instrumentos: Vías de ferrocarril en 1898 (ln km) Modelo: MCO en dos
  etapas (TSLS)
\item
  Variable de salida: ln de los kilómetros viajados por vehículo (VKT)
  Variable de interés: ln kilómetros de avenidas interestatales (IH)
  Instrumentos: Rutas de exploración en 1835 (ln km) Modelo: MCO en dos
  etapas (TSLS)
\item
  Restringido a la década de los 1980 Variable de salida: ln de los
  kilómetros viajados por vehículo (VKT) Variable de interés: ln
  kilómetros de avenidas interestatales (IH) Instrumentos: Vías de
  ferrocarril en 1898 (ln km) y rutas interestatales planeadas en 1947
  (ln km) Modelo: Máxima verosimilutud con restricciones de información
  (Limited information maximum likelihood - LIML)
\item
  Restringido a la década de los 1990 Variable de salida: ln de los
  kilómetros viajados por vehículo (VKT) Variable de interés: ln
  kilómetros de avenidas interestatales (IH) Instrumentos: Vías de
  ferrocarril en 1898 (ln km) y rutas interestatales planeadas en 1947
  (ln km) Modelo: Máxima verosimilutud con restricciones de información
  (Limited information maximum likelihood - LIML)
\item
  Restringido a la década de los 2000 Variable de salida: ln de los
  kilómetros viajados por vehículo (VKT) Variable de interés: ln
  kilómetros de avenidas interestatales (IH) Instrumentos: Vías de
  ferrocarril en 1898 (ln km) y rutas interestatales planeadas en 1947
  (ln km) Modelo: Máxima verosimilutud con restricciones de información
  (Limited information maximum likelihood - LIML)
\end{enumerate}

Modelos estimados:

M1: Regresión IV M2: M1 + Control por población M3: M2 + Controles de
geografía y Divisiones censales M4: M3 + Controles por características
socioeconómicas M5: M4 + Controles por el rezago de la población

Las regresiones de los módulos A, \ldots, E son realizadas con errores
robustos y clusters a nivel de MSA. Las regresiones de los módulos F, G
y H son realizadas con errores robustos.

\begin{itemize}
\tightlist
\item
  Contribución 9: Al igual que las salidas anteriores, esta también
  cuenta con una sección de arreglo de variables que será ignorada de la
  explicación en consideración que ya es recogida por los numerales
  anteriores.En este caso nuevamente se recurren a modelos panel, en los
  que se presentan diferentes especificaciones y métodos de estimación,
  a continuación, se presentará un resumen de cada uno de estas
  regresiones.
\end{itemize}

M1: Variable de salida: ln de los kilómetros viajados por vehículo (VKT)
Variable de interés: ln kilómetros de avenidas interestatales (IH) y ln
de la cantidad máxima de buses (pico) Modelo: MCO Panel con efectos
fijos de tiempo

M2: M1 + Controles por población M3: M2 + Controles por geografía y
unidades censales M4: M3 + Controles por variables socioeconómicas y
rezagos de la población M5: M1 + Efectos fijos por MSA M6: M2 + Efectos
fijos por MSA

M7: M1 Modelo: Máxima verosimilutud con restricciones de información
(Limited information maximum likelihood - LIML)

M8: M2 Modelo: Máxima verosimilutud con restricciones de información
(Limited information maximum likelihood - LIML)

M9: M3 Modelo: Máxima verosimilutud con restricciones de información
(Limited information maximum likelihood - LIML)

M10: M4 Modelo: Máxima verosimilutud con restricciones de información
(Limited information maximum likelihood - LIML)

Todas las regresiones cuentan con errores robustos y con errores
clusterizados a nivel de MSA en los casos en los que no se cuentan con
efectos fijos a nivel de MSA (5 y 6)

\begin{itemize}
\tightlist
\item
  Contribución 10:
\end{itemize}

Respecto a la preparación y reforma de la base de datos, las seccione
iniciales recogen este proceso. En esta sección se presentan 6 modelos
de regresión para comprobar la convergencia en promedio de anual de
tráfico diario (Annual average daily traffic AADT). A continuación, se
presentan las especificaciones de cada una de las regresiones.

M1: Variable de salida: Cambio AADT (delta AADT) Variable de interés:
Nivel inicial de AADT en avenidas interestatales Modelo: Mínimos
cuadrados ordinarios.

M2: M1 + Controles por población

M3: M2 + Controles por geografía y unidades censales

M4: M3 + Controles por variables socioeconómicas, participación inicial
de actividades de manufacturas y rezagos de la población

M5: M1 + Efectos fijos por MSA

M6: M3 + Control por participación inicial de actividades de
manufacturas Modelo: Máxima verosimilutud con restricciones de
información (Limited information maximum likelihood - LIML)

Todas las regresiones cuentan con errores robustos y con errores
clusterizados a nivel de MSA en los casos en los que no se cuentan con
efectos fijos a nivel de MSA (5).

\begin{itemize}
\tightlist
\item
  Contribución 11: Respecto a la preparación y reforma de la base de
  datos, las seccione iniciales recogen este proceso. En esta sección se
  presentan 10 modelos de regresión que recogen el comportamiento
  respecto a transporte de carga (Trucks). A continuación, se presenta
  cada una de las especificaciones presentadas.
\end{itemize}

M1: Variable de salida: ln VKT para transporte de carga (ln truck VKT)
Variable de interés: ln kilómetros de avenidas interestatales (IH)
Modelo: Mínimos cuadrados ordinarios.

M2: M1 + Controles por población

M3: M2 + Controles por geografía y unidades censales

M4: M3 + Controles por variables socioeconómicas

M5: M4 + controles por rezagos de la población

M6: M1 + Efectos fijos a nivel de MSA

M7: M2 + Efectos fijos a nivel de MSA

M8: M7 + Controles por variables socioeconómicas

M9: M2 Modelo: Máxima verosimilutud con restricciones de información
(Limited information maximum likelihood - LIML)

M10: M3 Modelo: Máxima verosimilutud con restricciones de información
(Limited information maximum likelihood - LIML)

Todas las regresiones cuentan con errores robustos y con errores
clusterizados a nivel de MSA en los casos en los que no se cuentan con
efectos fijos a nivel de MSA (6, 7 y 8).

\begin{itemize}
\tightlist
\item
  Contribución 12:
\end{itemize}

Al igual que las salidas anteriores esta cuenta con una parte de arreglo
de variables y modificación de la base de datos en forma (wide to long).
En esta tabla se presentan 15 modelos de regresión, divididos en 3
paneles con 5 especificaciones. A continuación, se hace un resumen de
estas:

Modelo, estimados mediante Mínimos cuadrados ordinarios en datos panel

Panel A: Variable de salida: ln VKT para las avenidas interestatales
(IH) únicamente para zonas urbanizadas Variable de interés: ln
kilómetros de avenidas interestatales urbanas (IHU), ln kilómetros de
avenidas interestatales no urbanas (IHNU) y ln kilómetros caminos
principales urbanos (IHU).

Panel B: Variable de salida: ln VKT para las avenidas interestatales
(IH) para zonas NO urbanizadas Variable de interés: ln kilómetros de
avenidas interestatales urbanas (IHU), ln kilómetros de avenidas
interestatales no urbanas (IHNU) y ln kilómetros caminos principales
urbanos (IHU).

Panel C: Variable de salida: ln VKT para los caminos principales (MRU)
únicamente para zonas urbanizadas Variable de interés: ln kilómetros de
avenidas interestatales urbanas (IHU), ln kilómetros de avenidas
interestatales no urbanas (IHNU) y ln kilómetros caminos principales
urbanos (IHU).

M1: Tiene en cuenta todas las variables de interés

M2: M1 + Controles por población

M3: M2 + Controles por geografía y unidades censales

M4: M3 + Controles por variables socioeconómicas

M5: M4 + controles por rezagos de la población

Todas las regresiones cuentan con errores robustos y con errores
clusterizados a nivel de MSA.

\hypertarget{results}{%
\section{Results}\label{results}}

En la tabla 1 describe los datos en general. Se observa que el promedio
de número carros que pasan por las autopistas de las MSA aumentaron de
4832 en 1983 a 9361 en 2003. También se muestra como aumentaron las
autopista en un 6\% durante 1983 y 1993 a 1993 y 2003.

\begin{center}
\smallskip\begin{large}Table 1 - Summary Statistics for Our Main Hpms\end{large}\\
and Public Transportation Variables\\
\smallskip
\begin{tabular}{lccc}
\hline \noalign{\smallskip} & 1983 & 1993 & 2003\\
\noalign{\smallskip}\hline \noalign{\smallskip}Mean daily VKT (IH, 1'000 km) & 7,776.63 & 11,904.95 & 15,960.58\\
\qquad Standard deviations & 16,623.98 & 24,251.06 & 31,579.29\\
Mean AADT (IH) & 4,832.08 & 7,174.15 & 9,360.78\\
\qquad Standard deviations & 2,726.30 & 3,413.23 & 4,091.54\\
Mean lane km (IH) & 1,140.27 & 1,208.16 & 1,279.75\\
\qquad Standard deviations & 1,649.76 & 1,729.43 & 1,857.58\\
Mean lane km (IH, per 10,000 population) & 1,140.27 & 1,208.16 & 1,279.75\\
\qquad Standard deviations & 1,649.76 & 1,729.43 & 1,857.58\\
Mean daily VKT (MRU, 1'000 km) & 14,553.36 & 22,449.55 & 31,242.38\\
\qquad Standard deviations & 36,303.49 & 49,132.38 & 70,691.90\\
Mean AADT (MRU) & 3,146.14 & 3,646.52 & 3,934.20\\
\qquad Standard deviations & 846.75 & 947.42 & 1,059.11\\
Mean lane km (MRU) & 3,884.81 & 5,071.38 & 6,471.45\\
\qquad Standard deviations & 7,925.68 & 9,118.73 & 12,426.76\\
Mean VKT share urbanized (IHU/IH) & 0.38 & 0.44 & 0.48\\
Mean lane km share urbanized (IHU/IH) & 0.30 & 0.36 & 0.40\\
Mean share truck AADT (IH) & 0.11 & 0.12 & 0.13\\
Peak service large buses per 10,000 population & 1.20 & 1.09 & 1.34\\
\qquad Standard deviations & 1.02 & 0.98 & 0.98\\
Peak service large buses & 168.88 & 165.26 & 217.16\\
\qquad Standard deviations & 562.93 & 561.72 & 741.98\\
Number MSAs & 228.00 & 228.00 & 228.00\\
Mean MSA population & 753,726.62 & 834,290.29 & 950,054.31\\
\noalign{\smallskip}\hline\end{tabular}\\
\begin{footnotesize}Notes: IH denotes interstate highways for the entire MSA. IHU denotes interstate highways for the urbanized areas within an MSA. MRU denotes major roads for the urbanized areas within an MSA.\end{footnotesize}\\
\smallskip
\end{center}



La tabla 2 reporta la elasticidad del trafico medido en VKT en los MSA
respecto al aumento de kilómetros de carreteras. Se destaca qeu
dependiendo de la década, la elasticidad pasa de 1.23 a 1.25. Sin
embargo, a medida que se analizan las carreteras menos urbanas, la
elasticidad cae.

\input{JN&CV_table_2.tex}

La tabla 3. Esta regresión que también es de elasticidad del trafico
respecto a vias, dado que la población y las variables socioeconómicas
estas correlacionas con variables no observables que aumentan el
tráfico, se incluyeron como controles. En esta especificación cae la
elasticidad a 0.71 y 0.94, dependeinte de la especificación usada y la
decada.

\input{JN&CV_table_3.tex}

La tabla 4, usa la especificación de la tabla 2 y de la tabla 3, pero
estimando un modelo panel con todos los años y para cada tipo de vias.

\input{JN&CV_table_4.tex}

La tabla 5, en lugar de estimar todos los años, estima la primera
diferencia para los valores de cada año. Entre los resultados
relevantes, se destaca que la importancia del cambio de la población se
incrementa.

\input{JN&CV_table_5.tex}

La tabla 6 es la más importante del paper, pues estima la elasticidad
del trafico respecto a vias instrumentando por diferentes controles.
Esta estimación se estima mediante IV. Se destaca que cuando se
utilizando todos los controles, la elastiidad es cercana a 1.

\input{JN&CV_table_6.tex}

La tabla 7 incluye la cantidad de buses, para intentar probar la
hióteiss de que el transporte público ayuda a disminuir el tráfico. Esta
tabla muestra que la elasticidad es menor a 1, mostrando que no tiene
efecto.

\input{JN&CV_table_7.tex}

La tabla 8, fue hecha intentando probar si el tráfico convergía un
punto. Para esto, se evidencia que durante 1980 y 200, la varianza del
total de carros en circulación disminuye.

\input{JN&CV_table_8.tex}

\input{JN&CV_table_9.tex}

\input{JN&CV_table_11.tex}

\end{document}
